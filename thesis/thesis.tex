%==========================================
%
% Academic thesis template
%
% Based on KOMA-Script book class.
%
% This template is not officially endorsed 
% by any educational institution.
%
% Ben Swift   22/6/12
% benjamin.j.swift@gmail.com
% http://github.com/benswift/thesis-template
% 
% This template is in the public domain
% 
%==========================================
% preamble
\documentclass[oneside,11pt,xetex]{scrbook}
\KOMAoptions{%
  headings=normal,
  captions=rightbeside,
  bibliography=totoc,
  listof=totoc}

%\usepackage{libertineotf}
\usepackage{amsmath}
\newcommand{\argmax}[1]{\underset{#1}{\operatorname{argmax}}}

\usepackage{fontspec}
\setmonofont[Scale=MatchLowercase,Mapping=tex-text]{Tahoma}

\usepackage{booktabs}
\usepackage{tabularx}
\usepackage{listings}
% if you want colour tables
% \usepackage{colortbl}

\usepackage[svgnames,hyperref]{xcolor}

\DeclareGraphicsExtensions{.pdf,.png,.jpg}
\graphicspath{{./figures/}}

\usepackage[margin=10pt,labelfont=bf]{caption}
\usepackage[labelformat=simple]{subcaption}
\renewcommand\thesubfigure{(\alph{subfigure})}

\usepackage{metalogo}
\usepackage{hologo}
\usepackage{verbatim}
\usepackage{setspace}
\usepackage{enumitem}

\newlist{transcriptlist}{description}{1}
\setlist[transcriptlist]{font=\sffamily\bfseries,
                              align=left,
                              leftmargin=1.6cm,
                              labelindent=\parindent, 
                              labelwidth=*}

\newenvironment{transcript}%
{\small\begin{transcriptlist}}%
{\end{transcriptlist}}

\newlist{headinglist}{description}{1}
\setlist[headinglist]{font=\sffamily\bfseries, 
                           leftmargin=0cm,
                           style=nextline}

\usepackage[%
backend=biber,
natbib=true,
backref=true,
citecounter=true,
dashed=false,
backrefstyle=three,
citestyle=authoryear-icomp,
firstinits=true,
maxcitenames=2,
maxbibnames=10,
uniquename=mininit,
bibstyle=authoryear,
url=false,
doi=false]{biblatex}

\AtEveryBibitem{\clearfield{month}}
\AtEveryCitekey{\clearfield{month}}

\nocite{*}
\addbibresource[datatype=bibtex]{thesis.bib}

\usepackage[english=british,threshold=15,thresholdtype=words]{csquotes}
\SetCiteCommand{\parencite}

\newenvironment*{smallquote}
  {\quote\small}
  {\endquote}
\SetBlockEnvironment{smallquote}

\usepackage[%
unicode=true,
hyperindex=true,
bookmarks=true,
pdftitle={Beyond .*Script},
pdfauthor={Veit Heller},
colorlinks=false,
pdfborder=0,
allcolors=DarkBlue,
pdfpagelabels,
hyperfootnotes=true]{hyperref}

\usepackage{bookmark}

\setcounter{tocdepth}{1}

\usepackage[%
acronym, 
nomain,
toc=true]{glossaries}

\usepackage{cleveref}

\newacronym{ast}{AST}{Abstract Syntax Tree}
\newacronym{ir}{IR}{Intermediate Representation}
\newacronym{ghc}{GHC}{Glasgow Haskell Compiler}
\newacronym{repl}{REPL}{Read-Eval-Print Loop}
\newacronym{ffi}{FFI}{Foreign Function Interface}
\newacronym{api}{API}{Application Programming Interface}
\newacronym{dom}{DOM}{Document Object Model}
\newacronym{w3c}{W3C}{World Wide Web Consortium}

\renewcommand{\thepage}{\roman{page}}

\makeindex
\makeglossaries


\usepackage{color}
\definecolor{lightgray}{rgb}{.9,.9,.9}
\definecolor{darkgray}{rgb}{.4,.4,.4}
\definecolor{purple}{rgb}{0.65, 0.12, 0.82}
\lstdefinelanguage{JavaScript}{
  keywords={break, case, catch, continue, debugger, default, delete, do, else, false, finally, for, function, if, in, instanceof, new, null, return, switch, this, throw, true, try, typeof, var, void, while, with},
  morecomment=[l]{//},
  morecomment=[s]{/*}{*/},
  morestring=[b]',
  morestring=[b]",
  ndkeywords={class, export, boolean, throw, implements, import, this},
  keywordstyle=\color{blue}\bfseries,
  ndkeywordstyle=\color{darkgray}\bfseries,
  identifierstyle=\color{black},
  commentstyle=\color{purple}\ttfamily,
  stringstyle=\color{red}\ttfamily,
  sensitive=true
}

\lstset{
   language=JavaScript,
   backgroundcolor=\color{lightgray},
   extendedchars=true,
   showstringspaces=false,
   basicstyle=\small,
   showspaces=false,
   numbers=left,
   numberstyle=\footnotesize,
   numbersep=9pt,
   tabsize=2,
   breaklines=true,
   showtabs=false,
   captionpos=b
}

\linespread{1.3}

\begin{document}


\title{Beyond .*Script}
\subtitle{Implementing A Language For The Web}

\author{Veit Heller}

\date{\today}

\publishers{%
  \normalsize{%
  A thesis submitted for the degree of \\
  B.Sc. of Applied Computer Science of \\
  The University of Applied Sciences Berlin}}

\uppertitleback{%
  \textbf{Institutional Address}\\
  HTW Berlin\\
  Campus Treskowallee\\
  Treskowallee 8\\
  10318 Berlin\\
  \textsc{Germany}\\
  \bigskip\\
  \textbf{Supervisory Panel}\\

  Prof. Hendrik Gärtner\\
  HTW Berlin\\

  Prof. Henrick Lochmann\\
  Prof. Henrik Lochmann\\
  HTW Berlin\\
  \bigskip\\
  Set with the help of {\KOMAScript} and
  \XeLaTeX.\\

  \copyright~\the\year. All rights reserved.}

\dedication{\small{\emph{For Meredith, Tobias and all the people who cope with me. Your undying support will not be forgotten.}}}

\pdfbookmark{Title Page}{Title Page}
\maketitle

\frontmatter

\vspace*{0.4\textheight}

\begin{center}
  Except where otherwise indicated, this thesis is my own original
  work.
\end{center}
\vspace*{4cm}

\begin{flushright}
  \begin{minipage}{4cm}
    Veit Heller\\
    \today
  \end{minipage}
\end{flushright}

\addchap{Abstract}

The modern web is comprised of an abundance of very different beasts. Technologies that powered
the first versions of the World Wide Web, such as HTML, CSS and JavaScript, and relatively new
conceptions like TypeScript, CoffeScript, PureScript, ClojureScript, Elm, LASS, SCSS, Jade and
Emscripten - to name but a few - are shaping the internet as we know it. There is one flaw that
many of the new technologies have in common, as different as they may look and feel - they are
mere preprocessors. In the end, it all boils down to the classic technologies again and we are
left with the same limited capabilities we have had for the last twenty years.

This thesis presents and evaluates a port of the zepto programming language to the web. It aims to
work as seamlessly with existing technologies as possible. It is largely influenced by R5RS Scheme.

\pdfbookmark{Contents}{Contents}
\tableofcontents

\printglossary[type=\acronymtype,title=Abbreviations]

\mainmatter

\pagestyle{headings}

\setchapterpreamble[u]{%
  \dictum[\emph{B. Kernighan}]{Controlling complexity is the essence of computer programming.}
  \bigskip}
\chapter{Introduction}
\label{chap:intro}

\section{Motivation}
\label{sec:Motivation}

JavaScript has, since its inception, attracted a lot of controversy. This is rooted
in various aspects of its design, from prototypal inheritance to operator precedence.
Prototypal inheritance has the reputation of being counter-intuitive, though it is older
than JavaScript, the first commonly known programming language that implements prototypal
objects being Self.


* things get better

* es 6 and es7 thank god

* a lot of research funneled into it

* still a fundamental rethinking might be necessary

NOTE: This had to be moved from chapter 2 to match the current design of the thesis. Please
excuse the lack of coherence to the part above.

\subsection{Lisp}

A common saying among programming language designers is that every programmer has written
their own implementation of Lisp. There are a lot of different implementations of Lisp
in the wild, even ones that compile to JavaScript\footnote{such as \href{https://github.com/clojure/clojurescript}{ClojureScript},
a backend of the Clojure compiler that targets JavaScript.}.

The main reason for that is often cited to be the simplicity of the language on a parsing
level. A simple Lisp can be implemented in less than one hundred lines of code, if no
intermediate representation is generated. This is made possible by the unique property
of Lisp of enclosing every statement in parentheses, where the first element within
those parentheses is the statement and the other elements are the arguments.
It can be evaluated straight from a textual level, because things such as operator precedence
and statement amiguity do not exist. In regular Lisp as specified in the initial paper by
John McCarthy\parencite{JCM} only six special forms exist to allow not only for Turing-completeness,
but also for expressiveness.

\subsection{zepto}

Zepto is a new Scheme implementation that aims to be as small as possible, to be able
to target a lot of different backends. Currently, LLVM and Erlang Core\footnote{Erlang Core 
is the \gls{ir} of Erlang code before it is complied. Resources and documentation
about it are sparse, it mostly seems to exist inside the BEAM's implementation.} bindings are
under development, the reference implementation is a simple interpreter that interprets code
directly from the \gls{ast}. This is slow but ensures a small interpreter size\footnote{The
entire codebase is only about 4000 lines of Haskell code.}. The compilers are written directly
in zepto itself.

The small code base makes zepto a good target for porting it to the web. Further, ecause it
is written in Haskell the code base was expected to be possibly almost entirely compilable
to JavaScript using GHC-JS, a backend for the \gls{ghc} targetting JavaScript instead of
native code. It offers many advanced features such as inlining of JavaScript into the code
base using a technique called quasi-quoting, where a special character sequence delimits the
inlined code, much like regular quotes. This tool set was expected to make the work of porting
an existing language to the web as simple as possible.

Of course there are other reasons to use a functional language as an example. With both syntax
and semantics differing wildly from JavaScript, this example makes way for languages more
closely related to JavaScript also making their eventual way into the browser.

\section{Goals of this Thesis}

The primary goal of this thesis is to present a novel approach at implementing languages
for the Web. This is exemplified by a sample implementation of a non-trivial functional programming
language.

* functional because different

* fairly different feature set

\section{Structure of this Thesis}

Chapter 2 examines related work in the field of cross-compilation into JavaScript and implementation
of interpreters that are directly embeddable into larger systems. This includes desktop applications,
game scripting engines and creative suites.

Chapter 3 gives an overview of the concept design and how the features are laid out to match the needs
of both the goals of this thesis and the prototype itself.

Chapter 4 presents the system design and how the prototype integrates into existing web components.

Chapter 5 discusses the implementation, picking out different fundamental parts of the system and
presents how they work.

Chapter 6 evaluates the prototype. This includes problems such as how well the integration of the
system worked and how it compares to the reference implementation of zepto.

Chapter 7 gives a short summary of what was done and gives an outlook to what might happen with
zepto, both the desktop and the JavaScript version, in the future.

\setchapterpreamble[u]{%
  \dictum[\emph{T. Peters}---The Zen of Python]{Practicality beats purity.}
  \bigskip}
\chapter{Related Work}
\label{chap:RelatedWork}

\section{Existing Projects}

\section{Existing Standards}


\setchapterpreamble[u]{%
  \dictum[\emph{T. Peters}---The Zen of Python]{Practicality beats purity.}
  \bigskip}
\chapter{Concept Design}
\label{chap:ConceptDesign}

\section{Construction Design}

\section{Additional Features}


\setchapterpreamble[u]{%
  \dictum[\emph{T. Peters}---The Zen of Python]{Practicality beats purity.}
  \bigskip}
\chapter{System Design}
\label{chap:System Design}

\section{Integration into the Web Ecosystem}


\setchapterpreamble[u]{%
  \dictum[\emph{Hofstadter’s Law}]{It always takes longer than you expect, even when you take into account Hofstadter’s Law.}
  \bigskip}
\chapter{Implementation}
\label{chap:Implementation}

The implementation philosophy philosophy of the port presented in this thesis has always been to
reuse as much code from the reference implementation as possible. This guided the flow of design
choises down a rather natural path and thus kept the implementation described here fairly short and
relatively trivial.

\section{Description of the Toolchain}

The tooling uses GHCJS, which is a backend for the \gls{ghc} compiler that targets JavaScript rather
than native code (TODO: remove this from introduction). This makes cross-compiling the code base to
JavaScript a rather simple undertaking.

* quasi-quoting

* jsbits

* as little dependence on it as possible

\section{Description of the Implementation}

As predicted in \ref{sec:Motivation}, the code base of zepto could be reused in almost its' entirety.
What had to be rewritten was mostly related to the startup of the interpreter, because the regular
paths into the code - either via a script being passed into it or launching an interactive
\gls{repl}\footnote{A \gls{repl} is an interactive code evaluation environment. Code is typed into
a prompt and immediately evaluated. The convenience of such a short feedback loop is often used
in the context of scripting languages and shells.} - were unavailable in the browser context.
Instead, a way of passing the sources from within \texttt{script} tags needed to be found. Further
customizations include a \gls{ffi} to enable better cross-evaluation of JavaScript and the adaptation
of existing \gls{api}s, such as the \gls{dom}.

\subsection{The \texttt{script} tag}

Initially, a \gls{dom} node walker was considered, but rejected relatively early because of two reasons:
Firstly, it introduced a layer of complexity from within JavaScript code that would have likely made
it brittle and hardly portable. Secondly, it would require a walk of the nodes every time a \gls{dom}
element is inserted or replaced, which is a common occurence in modern interactive web applications.

As of November of 2015, the \gls{w3c} specifies an \gls{api} that simplifies this process for the programmer.
Within their specification of the \href{https://www.w3.org/TR/dom/#mutationobserver}{DOM4}, the fourth
specification of \gls{api}s for the Web, an object called \texttt{MutationObserver} is included which
is able to register for \gls{dom} manipulations. Its main function will be triggered whenever a change
occurs within the \gls{dom} part that it registered for listening to.

This simplifies the implementation of a listener to DOM events a great deal. Only minimal programming
is required to configure the listener and to filter out all the nodes that are not \texttt{script} nodes
of the type \texttt{text/zepto}\footnote{This was chosen in analogy to the existing \texttt{text/javascript}
node type}.

A problem untended to with that method was nodes insert before the listener starts. This was resolved by
singling out all the \texttt{script} tags that are present before the listener starts and applying the
same filter/evaluation function to all of them. This also ensures that they are executed before any
additional code (and possibly dependent) is passed into the zepto object.

The code was then included in the \texttt{zepto} singleton, which is the global interpreter object
used for the management and interactivity of the zepto interpreter.

\begin{lstlisting}[language=JavaScript,caption=The final mutation observer code (simplified)]
// the initial observer and the function it takes
zepto.observer = new MutationObserver(zepto.handleMutation);

// this function will get a list of mutations and apply handleDom to them
zepto.handleMutation = function(mutations) {
  mutations.forEach(mutation => {
    mutation.addedNodes.map(zepto.handleDom);
  });
}

// evaluate if it is a text/zepto node
zepto.handleDom = function(node) {
  if (node.nodeName != "SCRIPT" || node.type != "text/zepto") {
    return null;
  }
  return zepto.eval(node.innerHTML);
}

// execute this on startup
window.onload = () => {
  let scripts = document.getElementsByTagName("script");
  scripts.map(zepto.handleDom);
}
// the extra arguments signify recursive listening
zepto.observer.observe(document, {childList: true, subtree: true});
\end{lstlisting}

\subsection{The \gls{ffi}}

The \gls{ffi} is a central part of the port. If it weren't usable, none of the browser's
capabilities could be used from within zepto, thus rendering the effort of bringing zepto into
the browser effectively useless. The \gls{api}s of the Web are a big part of what it means to
program for the browser, after all.

An initial sketch of the programming interface was extremely simplistic: a call to the
function \texttt{js} could be called with a string as argument, representing the textual
representation of the JavaScript program that should be run. It was piped to the JavaScript
function \texttt{eval} and the function returned an affirmative truth value. Quasi-quoting
larger blocks of JavaScript was also possible.

Of course this is unusable. The missing return value makes any effort of talking to an
\gls{api} impossible, as one could never yield any results. A different kind of return
value is needed.

The obvious but most challenging to implement solution would be to infer a fitting zepto
type for every return value in JavaScript and return a result depending on that. While
this could be seen as a rather elegant solution, it comes with its own set of caveats
and exceptions, as the mapping between JavaScript and zepto values is not always obvious.
A JavaScript object has too many properties that get lost in the process of translating
it to zepto as to make it intuitive.

\begin{lstlisting}[language=Lisp,caption=The ideal FFI]
; this would return an integer
(js "1 + 1")

; this would return a hashmap
(js "{key: \"val\""})

; this is problematic, because it will return an object
(js "new Error()")
\end{lstlisting}

Another problematic point is the implementation of JavaScript values in GHCJS. They
are opaque datatypes, aliases for addresses and byte vectors. While zepto supports
byte vectors and pointers, they are hardly a good representation for semantically rich
prototypes as they only offer a glance into the underlying implementation of the
JavaScript engine. While it is true that GHCJS itself provides methods for type
coercion, they are crude and possibly error-prone.

A simpler method that is still mostly sensible came up: returning the string
values of all of the values returned. While this places the burden of coercion
into the programmer's hand, it also gives them the power to make their own 
decisions of how to deserialize values. Functions for deserializing the most
common datatypes are included in the standard library of the JavaScript
implementation of zepto, to aid the programmer in the process of finding
the right methods of getting a value out of the \gls{ffi}.

This still does not solve the problem of helping manage classes, but it empowers
the programmer to find their own ways of serializing on the JavaScript side
and deserializing on the zepto side to preserve the information they need in
their specific programming context.

All of this needs an additional layer of abstraction to avoid unnecessary
boiler plate, but it is stable enough for most purposes that zepto in JavaScript
was used for yet.

\begin{lstlisting}[language=Lisp,caption=The final form of the FFI]
; the function string->number is a standard zepto function
(string->number (js "Math.pow(2, 32)"))

; this is an example of how to resolve the earlier problem:
; override the prototype of the object to return the value that is needed
(js "Error.prototype.toString = function() { return this.message; }")
(error (js "new Error(\"fatal error occured\")"))
\end{lstlisting}

Implementing the JavaScript to zepto \gls{ffi} was much simpler, as the
interpreter is defined within the JavaScript environment. A call to the
\texttt{eval} function of the \texttt{zepto} object with a string as argument
will return in the execution of this piece of code and the return the textual
representation of the zepto object so that the entire communication between
the languages is string-based.

\subsection{The \gls{dom}}

After building the \gls{ffi}, it was possible to implement the entire
communication with the \gls{dom} in terms of calls to foreign functions
and the parsing of their return values. This allows for a stable library,
because it is unintrusive and does not interfer with existing JavaScript
constructs.

Existing implementations often find it convenient to write a hybrid mix
of JavaScript and zepto code that calls each other at certain points.
This is, however, not advisable at the layer of libraries or utilities,
because it is at risk of getting in the way of the job.

\begin{lstlisting}[language=Lisp,caption=A minimal version of a DOM module]
; TODO: need to get this from work laptop
(module "dom"
  (export
   `("add-node" add-node)
   `("get-elem" get-elem))

  (define (add-node node)
    ())

  (define (get-elem elem)
    ()))
\end{lstlisting}

\clearpage

\setchapterpreamble[u]{%
  \dictum[\emph{R. Buckminster Fuller}]{When I'm working on a problem, I never think about beauty. I think only how to solve the problem. But when I have finished, if the solution is not beautiful, I know it is wrong.}
  \bigskip}
\chapter{Evaluation of the Prototype}
\label{chap:evaluation}

\section{Seamlessness of Integration}

\section{Test Against Standard Implementation of Zepto}

* added: ffi

* removed load statement, REPL functionality

* inserting libraries?

\setchapterpreamble[u]{%
  \dictum[\emph{R. Buckminster Fuller}]{When I'm working on a problem, I never think about beauty. I think only how to solve the problem. But when I have finished, if the solution is not beautiful, I know it is wrong.}
  \bigskip}
\chapter{Summary and Outlook}
\label{chap:outlook}

* porting efforts

* compiler efforts

* zeps

* classes

\chapter{Conclusion}
\label{chap:conclusion}

* fixing the web not necessary

* rethinking it possible


%============================================
%============================================
% end matter

\backmatter

\bookmarksetup{startatroot}

\printbibliography[title=References,heading=bibintoc]

\end{document}
